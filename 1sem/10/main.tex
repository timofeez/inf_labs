\documentclass{article}

\usepackage[russian]{babel}
\usepackage[utf8]{inputenc}
\usepackage{geometry}
\usepackage{mathtools}
\usepackage{titlesec}
\usepackage{graphicx}
\usepackage{enumitem}
\usepackage{amssymb}
\usepackage{subcaption}

\geometry{paperwidth=491.84pt,paperheight=9.25in}
\geometry{lmargin=86.383pt,bmargin=80pt,rmargin=86.383pt,tmargin=80.55pt}
\setlength{\parindent}{0cm}

\title{Page 316}
\author{Григорьев Т. А.}
\date{December 2023}

\begin{document}
\pagenumbering{\textsl{316}}

\textsf{\textbf{11.2. Теоремы Ролля, Лагранжа и Коши \\ о средних значениях}}
\\

\textbf{\textsc{ТЕОРЕМА 2}}
\textsf{\textbf{(теорема Ролля\footnote{М. Ролль (1652—1719) — французский математик.})}}
\textit{Пусть функция $f$:}
\begin{enumerate}[wide, labelwidth=2pt, labelindent=0pt, topsep=0pt, itemsep=0pt]
    \item[1)] \textit{непрерывна на отрезке $[a, b]$;}
    \item[2)] \textit{имеет в каждой точке интервала $(a, b)$ конечную или определенного знака бесконечную производную;}
    \item[3)] \textit{принимает равные значения на концах отрезка, т. е. $f(a) = f(b)$.}
\end{enumerate}
{\setlength{\parindent}{1cm}


\textit{Тогда существует хотя бы одна такая точка $\xi, a < \xi < b$, что $f'(\xi) = 0$.}


Геометрический смысл теоремы Ролля состоит в том, что на графике функции, удовлетворяющей условиям теоремы Ролля, имеется по крайней мере одна точка, в которой касательная горизонтальна (рис. 51).

\setlength{\parindent}{0cm}
\textsc{Доказательство.} Если для любой точки $x$ интервала $(a, b)$ выполняется равенство $f(x) = f(a) = f(b)$, то функция $f$ является постоянной на этом интервале и поэтому для любой точки $\xi \in (a, b)$ выполняется условие $f'(\xi) = 0$.
{\setlength{\parindent}{1cm}

Пусть существует точка $x_0 \in (a, b)$, для которой $f(x_0) \ne f(a)$, например, $f(x_0) > f(a)$. Согласно теореме Вайерштрасса о достижимости непрерывной на отрезке функцией своих наибольшего и наименьшего значений (см. теорему 1 в п. 6.1), существует такая точка $\xi \in [a, b]$, в которой функция $f$ принимает наибольшее значение. Тогда
\begin{equation*}
    f(\xi) \ge f(x_0) > f(a) = f(b).
\end{equation*}
Поэтому $\xi \ne a$ и $\xi \ne b$, т. е. точка $\xi$ принадлежит интервалу $(a, b)$ и функция $f$ принимает в ней наибольшее значение. Следовательно, согласно теореме Ферма (см. теорему 1 в п. 11.1) выполняется равенство $f'(\xi) = 0$. $\Box$
\begin{figure}[h]
    \includegraphics[scale=0.8]{plot.png}
  \begin{subfigure}{0.5\textwidth}
    Заметим, что все предпосылки теоремы Ролля существенны. Чтобы в этом убедиться, достаточно привести примеры функций, для которых выполнялись бы два из трех условий теоремы, третье уже не выполнялось бы и у которых не существует точки $\xi$
  \end{subfigure}
\end{figure}
\end{document}
